\documentclass[11pt]{article}

\newcommand{\version}{1.0.3}

\title{The AtomDB Charge Exchange Model}
\author{Randall Smith, Adam Foster}
\date{April 8, 2014}

\begin{document}
\maketitle

The ACX package includes two primary tools: the XSPEC package ACX, which
includes a number of related charge exchange models described below,
and a program `dacx', which displays line strengths from ACX models
based on parameters identical to those in the XSPEC models.  

\section*{Version History}
\subsection*{April 8th 2014: version 1.0.0}
Initial Release
\subsection*{June 26th 2015: version 1.0.1}
Bugfix for \texttt{dacx}.
\subsection*{July 18th 2018: version 1.0.2}
Updated to work with HEASOFT 6.24, bugfix for \texttt{acxion} model.
\subsection*{April 4th 2019: version 1.0.3}
Further bugfix for \texttt{acxion} model to prevent crashes in certain conditions.


\section*{Intallation}

The acx package makes an honest, but limited, attempt to be installable
on multiple platforms.  It should work on any modern Linux system and
on Macs running OSX 10.6 or later.  It may work on other machines; if
it fails, however, please consider how much you paid for this before
complaining too loudly.  Complaining politely, however, may prove
useful, especially if you're willing to work with us to see how to fix
it.

Since the primary purpose of ACX is to create a model usable in XSPEC,
it is important to have XSPEC (and, in fact, the entire HEASOFT
package) available on the computer you're going to install it on.
HEASOFT is available at {\tt
http://heasarc.gsfc.nasa.gov/docs/software/lheasoft/}, in case you do
not have it already installed.  If you plan to use acx, I strongly
recommend you use the `build-from-source' option, as you're going to
want to make sure the C compiler you used to build HEASOFT is the same
as the one you built acx with.  This may not be absolutely necessary,
but I've found that the XSPEC model creation tool is a bit fragile so
it's best to humor it.  The following procedure should work so long as
you have HEASOFT installed and ready when you run the compilation
step.  That means you should be able to type `{\tt xspec}' at the
command line and have xspec start {\bf before}\ you begin the
compilation process.  If you don't, acx will {\bf not}\ compile the
xspec module.  The installation procedure is as follows, for version
\version:

{\tt
\noindent unix$>$ source \$HEADAS/headas-init.csh \\
\noindent unix$>$ tar zxf acx-\version.tar.gz \\
\noindent unix$>$ cd acx-\version \\
\noindent unix$>$ ./configure \\
\noindent unix$>$ make \\
}

This should compile all the necessary libraries, along with the dacx
code (in the src/ directory) and the XSPEC models in the xspec/
directory.  Note that the installation is done in place; if you really
want to use the `{\tt make install}' option, you can try it but there
are no guarantees.

If the above does {\bf not}\ work and all you really want is the xspec
module, I recommend restarting from the beginning and just trying to
compile the XSPEC module thusly:

{\tt
\noindent unix$>$ source \$HEADAS/headas-init.csh \\
\noindent unix$>$ tar zxf acx-\version.tar.gz\\
\noindent unix$>$ cd acx-\version/xspec\\
\noindent unix$>$ cp acx.h.in acx.h\\
\noindent \dots edit acx.h to put in the full path to the acx-\version
directory \dots\\
\noindent unix$>$ initpackage acx model.dat .\\
\noindent unix$>$ hmake\\
}

This will compile just the XSPEC model.  It's important that when you
edit the {\tt acx.h}\ file you define the DIRECTORY variable to just
point to the top level directory, ending in acx-\version, not the
data/ subdirectory.  The code tacks on the /data/ part itself.

\section*{The XSPEC module}

\subsection*{Installing into XSPEC}

If the installation worked out ok, you should now be able to install
the acx model into your running copy of XSPEC.  acx requires a fairly
large chunk of memory to install, and seems to work best if you load
it right after starting XSPEC.  Not `working best' translates into
unexpected core dumps, losing all of your fits, so I urge you to
install acx immediately upon starting XSPEC if you plan to use it.

Installing acx into XSPEC is simple:

\noindent {\tt XSPEC$>$\ lmod acx /path/to/acx-\version/xspec}

If this completes without an error message, you should be ready to go.

\subsection*{The basic acx / vacx models}

For more information about the physics in the acx models, please see
the Smith et al.\ (2014, ApJ) and Smith et al. (2012, AN) papers in
this directory.  These 
instructions provide only a how-to guide.  The {\tt acx}\ model takes
seven parameters, defined here:
\begin{description}
\item[kT] The equilibrium ion population distribution ``temperature'',
  in keV.  Note that this does not a true Maxwellian velocity
  distribution; instead, it sets the ion population as though it were
  in collisional ionization equilibrium created by electrons at this
  temperature.  In practice, ion population distributions involved in
  charge exchange may not be represented by a single temperature or
  even multiple temperatures.  However, this is a necessary
  simplification to make an XSPEC model practicable.  
\item[FracHe0] The fraction of neutral Helium (relative to the total
  neutral population, assumed to be H and He) in the plasma.  By
  default set to a cosmic value of 1/10th He, or 0.090909.
\item[Abundanc] The relative (to Anders \& Greveese 1989 solar values)
  abundance of metals in the plasma.  {\bf Note:} This should be kept
  frozen in almost all cases, because a change in the overall metal
  abundance will act as a change in the normalization value.  In a
  pure charge exchange spectrum there is no way to measure the
  absolute abundances since Hydrogen, the typical astronomical
  normalizing abundance, does not create any detectable charge
  exchange emission.  Users strongly urged to ensure they understand
  why this is so before using acx.
\item[redshift] The redshift of the emitting plasma.  
\item[swcx] For most cases, this should be 0; set it to 1 for modeling
  solar wind charge exchange (SWCX).  See \S\ref{subsec:swcx} for more details.
\item[model] The assumed model for the distribution of $n$\ and $l$\
  in the charge exchanging ions.  See \S\ref{subsec:model} for more details.
\item[norm] A scaling for the emissivity of the plasma.  In concept,
  this should depend upon the densities of the charge exchanging ions
  and neutrals and their distance from Earth.  However, the acx model
  is an approximation that does not include the actual cross section
  of the charge exchange process itself, and so the norm has no
  physical use except as a relative scaling.  Users strongly urged to
  ensure they understand what this means before using acx.
\end{description}

The corresponding variable-abundance version {\tt vacx}\ takes similar
parameters except it allows for the relative abundances of individual
elements to be varied.  At least one of these elements should be
frozen, however, to avoid problematic interactions with the XSPEC
norm.

\subsection*{The swcx flag\label{subsec:swcx}}

The sole difference between the swcx=0 and the swcx=1 models is in how
the charge exchange model is implemented.  In the standard model,
suitable for sources galactic or extragalactic sources, the
assumption is made that once an ion interacts via charge exchange with
a neutral atom, it will then immediately find {\sl another}\ neutral
and will repeat the charge exchange process until the ion is fully
neutralized.  Thus a fully-stripped O$^{+8}$\ ion will emit not only O
VIII lines, it will also emit O VII, O VI, and on down until the
oxygen can no longer undergo charge exchange.  Physically, the picture
is that of a hot ion from some source -- a supernova shock, galactic
superwind, etc -- has run into a dense neutral cloud.  In this case,
the CX cross section is so large that the process will neutralize the
ion in a very short distance, certainly smaller than can be resolved
with an X-ray telescope.  

The only time this statement is not true is if we are actually within
the cloud of ions -- in this case, coming from the solar wind and
interacting with neutral matter in the heliosphere.  In this case,
it's entirely possible the ion will not interact again within the
field of view of the telescope.  One way to imagine this process is to
consider following the progress of a large coronal mass ejection (CME)
as it travels through the heliosphere.  At early times, while the CME
is just leaving the Sun, it will have mostly highly-ionized gas, so
any CX will only show high ions.  After it crosses one AU and heads
out into the more distant reaches of the solar system, it will show
lower ions with a lower temperature.  This would be clearly require
swcx=1, as just because an O$^{+8}$\ ion undergoes CX and emits a O
VIII photon does not imply we will also see an equal number of O VII
and O VI photons in the same observation.  Instead, a O$^{+7}$\ ion
will travel further from the Sun and eventually undergo more CX
reactions.  

As a practical matter, then, these models will have far less emission
for a given temperature and normalization value than the standard
models.

\subsection*{The acxion model}

This model is useful largely as a diagnostic of charge exchange
patterns from particular ions.  The model takes 5 parameters, listed
here:

\begin{description}
\item[FracHe0]  Same as the acx model.
\item[El] Z for the ion, or the number of protons in the ion
\item[rmJ] The ionization stage; 0 for neutral, Z for fully-stripped.
\item[redshift] Same as the acx model.
\item[model]Same as the acx model.
\item[norm]Same as the acx model.
\end{description}

Thus, to find out what the spectral shape of charge exchange emission
from Ne$^{10}$\ is, use El=10, rmJ=10.

\subsection*{The meaning of the 'model' parameter\label{subsec:model}}

The model parameter sets how the code assumes electrons from the
neutral participant in the charge exchange land on the ion.  
Essentially, there are two primary questions: which (1) principal
quantum number $n$\ state and (2) total angular momentum $l$\ state
will be populated?  The spin state could also vary (and does, in
some cases, as a function of impact velocity), but we assume this
distribution is done equally in all cases.  This latter assumption is
based on a lack of available data, however, and not out of any
physical principle.    

There are 16 different models included in the initial distribution.
These come about as we have two different versions of how the $n$\
state is distributed, and eight different cases for $l$.

Predicting the actual line emission following charge exchange thus requires
first determining the exact atomic level (or distribution of levels)
of the charge-exchanged ion.  We follow the approximation described by
Janev \& Winter (1985), who found that the peak of the principal
quantum number $n$\ distribution is at
\begin{equation}
n' = q \sqrt{{{I_H}\over{I_p}}}\Big(1 + {{q-1}\over{\sqrt{2q}}}\Big)^{-1/2}
\end{equation}
where again $q$\ is the charge of the ion, $I_H$\ is the ionization
energy of the neutral ion (assumed here to be hydrogen), and $I_p$\ is
the ionization potential in atomic units.  In the case of the ACX
model parameter, models 1-4 and 9-12 assume all the ions ended up in
this level, while models 5-8 and 13-16 use a weighted distribution
between, so that if $n'$\ is equal to (say) 4.7, then 30\% of the ions
would populate $n=4$\ while 70\% would be in $n=5$.  In practice,
then, models 5-8 and 13-16 should be a more accurate depiction of the
real distribution, although a comparison between related models ({\it
  e.g.} 3 \& 7, or 4 \& 8) shows that there is relatively little
difference in the final spectrum in practice.

The primary uncertainty in the process, however, is the angular
momentum ($l$) of the exchanged electron. The correct result
will be velocity-dependent, which is problematic since in most cases
the input ion velocity (or position) will not be known. Following the
approximate nature of this model, we address this uncertainty by
simply providing a range of options for the model.  

By default we use orbital angular momentum distributions based on the
$nl$\ of the captured electron, creating the 1-8 model cases.  This approach
best handles the intermediate weight ions where $LS$\ coupling is
inappropriate and $L$\ is not a reliable quantum number.  However, it
does present a different set of problems with heavier ions where there
is significant configuration mixing. For example, defining which
levels truly represent a captured $11f$\ electron is not exact.

For reasons of convenience, we initally developed methods that
replaced the orbital angular momentum $l$\ with the total orbital
angular momentum $L$. This was used for the fits presented in Smith et
al (2012, 2014) because there is no practical difference between the
two for the Li-, He- and H-like ions which dominate X-ray spectra, and
$L$\ is stored explicitly in AtomDB.  These no longer our default
models, however, and are now labeled the 9-16 models.  They may be
removed in future releases.

Given the qualitative nature of these distributions, we currently
include both approaches in the distributed ACX model. Regardless of
these differences, within all the levels which can be created by each
capture with varying $L$, $S$, and $J$\ quantum numbers, we distribute
the population using statistical weighting $(2J+1)$.

The four different distributions considered based on either orbital
(1-8) or total (9-16) angular momemtum are:
\begin{enumerate}
\item ``Even:" weighted evenly by angular momentum, models 1,
  5, 9, and 13.
\item ``Statistical:" weighted by the relative statistical weight of
  each level, models 2, 6, 10, and 14.
\item ``Landau-Zener:'', models 3, 7, 11, and 15, weighted by the function 
\begin{equation}
W(l)={{l(l+1)(2l+1)\times(n-1)! \times (n-2)!}\over{ (n+l)! \times (n-l-1)!}}
\end{equation}
\item ``Separable:'', models 4, 8, 12, and 16, weighted by the function 
\begin{equation}
W(l) = {{(2l+1)}\over{Z}}\times \exp\Big[{{-l \times(l+1)}\over{z}}\Big]
\end{equation}
\end{enumerate}

The latter two methods in this list are from Janev \& Winter
(1985). The separable is the default, preferably with the second
$n$\ distribution model using the orbital angular approach, making the
standard starting method model 8.  We hope that providing the
alternative models we will allow users to test the sensitivity of
their data to the model approximation. Despite the simple nature of
these models, we expect they will be useful to check if charge
exchange could or could not be responsible for some or all of an
observed spectrum.

For convenience, Table~\ref{tab:model} lists each case and the
assumptions involved.  For slow winds ($v < 1000$\,km/s), either model
7 or 8 is recommended, along with tests for sensitivity using other
values.  The statistical model would only be applicable for
high-velocity CX, while the even distribution would be a good test for
sensitivity. 

\section*{The dacx program}

The dacx program is designed to output the line strengths from a
charge exchange model using the values obtained from an XSPEC fit.
The code outputs the line id's, their wavelengths, and the line
strengths in units of photons cm$^{-2}$s$^{-1}$.  It uses the
ever-popular IRAF interface, which should be familiar to all HEASOFT
users via the {\tt pset}, {\it punlearn}\ and other commands.  To run
dacx, first set the environment variable PFILES to ../pfiles and then
just type dacx:

{\tt
\begin{verbatim}
unix> setenv PFILES ../pfiles
unix> bin/dacx
\end{verbatim}
}

The code will prompt you for all the necessary options, which are for
convenience also listed here:

\begin{description}
\item[OutputFileName] The name of the optional output file; set to
  STDOUT if you want screen output
\item[Wavelength] Set this to yes if you want to work in wavelength
  units (=\AA).  Otherwise, the code assumes Energy units (=keV)" 
\item[LambdaMin]  The minimum wavelength (or energy) to output
\item[LambdaMax]  The maximum wavelength (or energy) to output
\item[MinEmiss] The minimum emissivity (in photons cm$^{-2}$s$^{-1}$)
  to still output
\item[kT] Equilibrium ion balance temperature
\item[FracHe] Fraction by number of Neutral He
\item[Abundance] Abundance relative to solar (Anders and Grevesse)
\item[Model] Model number (1-8)
\item[Norm] XSPEC Normalization
\item[redshift] Redshift (hidden; must be actively set with pset )
\item[swcx] Standard model (if no) or Solar Wind Charge Exchange (if yes).  (hidden; must be actively set with pset )
\item[C] Relative abundance of Carbon (hidden; must be actively set with pset )
\item[N] Relative abundance of Nitrogen (hidden; must be actively set with pset )
\item[O,]Relative abundance of Oxygen (hidden; must be actively set with pset )
\item[Ne]Relative abundance of Neon (hidden; must be actively set with pset )
\item[Mg]Relative abundance of Magnesium (hidden; must be actively set with pset )
\item[Al]Relative abundance of Aluminum (hidden; must be actively set with pset )
\item[Si]Relative abundance of Silicon (hidden; must be actively set with pset )
\item[S ]Relative abundance of Sulfur (hidden; must be actively set with pset )
\item[Ar]Relative abundance of Argon(hidden; must be actively set with pset )
\item[Ca]Relative abundance of Calcium (hidden; must be actively set with pset )
\item[Fe]Relative abundance of Iron (hidden; must be actively set with pset )
\item[Ni]Relative abundance of Nickel (hidden; must be actively set with pset )
\item[clobber] If yes, can overwrite existing output file. (hidden; must be actively set with pset )
\end{description}

\pagebreak 

\begin{table}
\caption{Meanings of the (v)acx model value\label{tab:model}}
\begin{tabular}{ll}
\hline \hline Value & Assumptions \\ \hline
1 &All CX into one $n$\ shell, even distribution by orbital angular momentum $l$. \\
2&All CX into one $n$\ shell, statistical distribution by orbital angular momentum $l$. \\
3&All CX into one $n$\ shell, Landau-Zener distribution by orbital angular momentum $l$. \\
4&All CX into one $n$\ shell, Separable distribution by orbital angular momentum $l$. \\  
5&CX distributed by weight into neighboring $n$\ shells, even
distribution \\
 & by orbital angular momentum $l$. \\     
6&CX distributed by weight into neighboring $n$\ shells, statistical
distribution \\
 & by orbital angular momentum $l$. \\
7&CX distributed by weight into neighboring $n$\ shells, Landau-Zener
distribution \\
 & by orbital angular momentum $l$.\\
8&CX distributed by weight into neighboring $n$\ shells, Separable
distribution \\
 & by orbital angular momentum $l$. \\ \hline
9 &All CX into one $n$\ shell, even distribution by total angular momentum $L$. \\
10&All CX into one $n$\ shell, statistical distribution by total angular momentum $L$. \\
11&All CX into one $n$\ shell, Landau-Zener distribution by total angular momentum $L$. \\
12&All CX into one $n$\ shell, Separable distribution by total angular momentum $L$. \\
13&CX distributed by weight into neighboring $n$\ shells, even
distribution \\
 & by total angular momentum $L$. \\
14 &CX distributed by weight into neighboring $n$\ shells, statistical
distribution \\ 
 & by total angular momentum $L$. \\
15&CX distributed by weight into neighboring $n$\ shells, Landau-Zener
distribution \\
& by total angular momentum $L$. \\
16&CX distributed by weight into neighboring $n$\ shells, Separable
distribution \\
 & by total angular momentum $L$. \\
\end{tabular}
\end{table}

\end{document}
